\documentclass[a4paper,10pt]{scrartcl}[2003/01/01]
\usepackage[ngerman]{babel}
\usepackage[T1]{fontenc}
\usepackage{inputenc}
\usepackage{graphicx}
\usepackage{listings}
\usepackage{enumerate}
\usepackage{verbatim}
\title{Software Architectures}
\subtitle{Exercise 4}
\author{ Felix Baumann \\ Manuel Gottschlich \\  Alexey Gy\"ori 352678 \\ Vincent Wehrwein \\ Markus Weller 352466}
\begin{document}
    \maketitle
    \section*{Aufgabe 4.1}
Blub
    \section*{Aufgabe 4.2}
Variant programming uses superclass method outputs to do further processing in a sub class. E.g. We have a superclass Car and a subclass Porsche. Both superclass and subclass have a method to calculate the horse power output. This method inside Porsche is defined in a way s.t. it uses the code of the same method of the superclass and only includes the code differences.

\begin{verbatim}
module body Car 
 function CalcHorsePower() 
 	return 20;
 end CalcHorsePower;
end Car;

module Porsche extends Car 
 function CalcHorsePower()
  return super.CalcHorsePower() * 1.5;
 end CalcHorsePower;
end Porsche;
\end{verbatim}
    \section*{Aufgabe 4.3}
    
    \textbf{Subsytem definition}
\begin{quote}
A subsystem is a set of logically related modules placed in a new large block. It is determined which modules create the interface of the subsystem. The internal subarchitecture is hidden.
   
\end{quote}
   
    \section*{Aufgabe 4.4}
Miep bieb

\end{document}