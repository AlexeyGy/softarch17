	
\documentclass[a4paper,10pt]{scrartcl}[2003/01/01]
\usepackage[ngerman]{babel}
\usepackage[T1]{fontenc}
\usepackage{inputenc}
\usepackage{graphicx}
\usepackage{enumitem}
\usepackage{listings}

\title{Software Architectures}
\subtitle{Exercise 2}
\author{ Felix Baumann \\ Manuel Gottschlich \\  Alexey Gy\"ori 352678 \\ Vincent Wehrwein \\ Markus Weller}
\begin{document}
\maketitle

\section*{Aufgabe 2.1}
\subsection*{a) Abstract data type}
In some cases, more than one instances of very similar objects are required. An  entertainment database for example can hold a collection of movies, a collections of books and a collection of games. The similarities of these collections can be implemented in an abstract data type where all three collections are derived. Abstract means that the data type is not fixed to one kind of object but can take different forms like in this example movies, books and games. The motivation behind abstract data types is that we can reuse existing implementations in different situations. If we had implemented movies, books and games as abstract data objects in this example we would have had a lot of code duplication and the program would be difficult to maintain.

\subsubsection*{Example 1: functional language}
\lstinputlisting{Exa)1.txt}

\subsubsection*{Example 2: OO language}
\lstinputlisting{Exa)2.txt}


\subsection*{b)}
An Abstract data object is a data structure which holds and manages data. The implementation of the ado is hidden in the body, so other parts of the program can use it without depending of the implementation of the module. The function used to access and process the data are defined in the interface of the ado. Besides them ado often have security functions to check if an operation is legal, for example if a stack is not empty. 

\subsubsection*{Example 1: OO language}
The following ado in Java holds data about a movie and can print an information about the movie. 
\lstinputlisting{Exb)1.txt}

\subsubsection*{Example 2: functional language}
The following Ado is a simple counter in C which starts by 0 and counts up, when the function countUp() is called. The interface consists of void countUp() and int getValue().
\lstinputlisting{Exb)2.txt}


\section*{Aufgabe 2.2}

\section*{Aufgabe 2.3}


\end{document}